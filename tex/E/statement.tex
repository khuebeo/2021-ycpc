$X$ is a positive number and $P$ is a set of potrysitive numbers. 
Compute the total number of combinations of the $P$'s elements 
such that the sum of the elements of each combination is equal to $X$. 
The elements of $P$ can be reused.

The followings are three examples.
Example One: $X = 6, P = \{2, 3, 7\}$. 
The number of combinations is $2$. 
The combinations are: 1) $\{2,2,2\}$; and 2) $\{3,3\}$.

Example Two: $X = 7, P = \{2, 7, 3\}$. 
The number of combinations is $2$. 
The combinations are: 1) $\{2,3,2\}$; and 2) $\{7\}$.

Example Three: $X = 9, P = \{2, 3, 7, 3\}$. 
The number of combinations is $3$. 
The combinations are: 1) $\{2, 7\}$; 2) $\{2, 2, 2, 3\}$; and 3) $\{3, 3, 3\}$.
